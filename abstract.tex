% % Aus der Promotionsverordnung MNF §11 (4):
% % "Dem wissenschaftlichen Teil der Dissertation ist eine allgemein
% % verständliche kurze Zusammenfassung voranzustellen."

% \cleardoublepage
% \pdfbookmark{Zusammenfassung}{abstract_DE}

% { % need this extra grouping to avoid carry-over effects
% \begin{otherlanguage}{ngerman}
% \enlargethispage{4\baselineskip}
% \begin{center}\abstractnamefont
%   Lorem ipsum dolor sit amet, consectetur adipiscing elit, sed do eiusmod tempor
%   incididunt ut labore et dolore magna aliqua. Ut enim ad minim veniam, quis
%   nostrud exercitation ullamco laboris nisi ut aliquip ex ea commodo consequat.
%   Duis aute irure dolor in reprehenderit in voluptate velit esse cillum dolore
%   eu fugiat nulla pariatur. Excepteur sint occaecat cupidatat non proident, sunt
%   in culpa qui officia deserunt mollit anim id est laborum. Lorem ipsum dolor
%   sit amet, consectetur adipiscing elit, sed do eiusmod tempor incididunt ut
%   labore et dolore magna aliqua. Ut enim ad minim veniam, quis nostrud
%   exercitation ullamco laboris nisi ut aliquip ex ea commodo consequat. Duis
%   aute irure dolor in reprehenderit in voluptate velit esse cillum dolore eu
%   fugiat nulla pariatur. Excepteur sint occaecat cupidatat non proident, sunt in
%   culpa qui officia deserunt mollit anim id est laborum. Lorem ipsum dolor sit
%   amet, consectetur adipiscing elit, sed do eiusmod tempor incididunt ut labore
%   et dolore magna aliqua. Ut enim ad minim veniam, quis nostrud exercitation
%   ullamco laboris nisi ut aliquip ex ea commodo consequat. Duis aute irure dolor
%   in reprehenderit in voluptate velit esse cillum dolore eu fugiat nulla
%   pariatur. Excepteur sint occaecat cupidatat non proident, sunt in culpa qui
%   officia deserunt mollit anim id est laborum.

% \end{center}
% \end{otherlanguage}
% }


%%%%%%%%%%%%%%%%%%%%%%%%%%%%%%%%%%%%%%%%%%%%%%%%%%%%%%%%%%%%%%%%%%%%%%%%

\cleardoublepage
\pdfbookmark{Abstract}{abstract_EN}

\enlargethispage{4\baselineskip}
\begin{center}\abstractnamefont
  \textbf{Abstract}
\end{center}

The fact that a scientific finding can be replicated in an independent
replication study is central for its credibility. However, various large-scale
replication failures have shown that the replicability of scientific findings is
often lower than expected. This ``replication crisis'' has led to several
methodological reforms in the past decade, an increased conduct of replication
studies being one of them. Yet despite this rise of replication research, there
is no consensus on which statistical methods should be used for the design and
analysis of replication studies; Various methods already exist and various new
ones have been proposed in response to the crisis. This thesis centres around
the proposal from \citet{Held2020}, which is based on a reverse-Bayes approach.
The key idea is to reverse the traditional ``forward'' use of Bayes' theorem
(``prior + likelihood $\rightarrow$ posterior''), starting instead with a
pre-specified posterior and deducing the corresponding prior (``posterior +
likelihood $\rightarrow$
prior''). % As such, the reverse-Bayes approach naturally fits to the replication
% setting as it enables the mathematical formalization of scepticism regarding the
% finding from the original study.
This allows to challenge the original finding by determining a sceptical prior
for the underlying effect size such that the resulting posterior is no longer
convincing. This prior then represents the position of a sceptic who remains
unconvinced by the original study. Whether or not the sceptics' position is
justified can then be assessed in light of the new data from the replication
study. The larger the conflict between the replication data and the sceptic, the
larger the degree of replication success. This procedure can be summarized in a
single quantitative measure of replication success, termed the sceptical
$p$-value.

The first and largest part of the thesis consists of extending this procedure. A
first extension replaces tail probabilities by Bayes factors as measures of
evidence. The sceptical prior is now determined such that the original finding
is no longer convincing in terms of a Bayes factor. In contrast to tail
probabilities, Bayes factors have a more natural interpretation and allow for
direct quantification of evidence for one hypothesis versus another. Similarly
as with the sceptical $p$-value, the procedure leads to a single measure
quantifying the degree of replication success called the sceptical Bayes factor.
A second extensions recalibrates the original procedure to produce more
appropriate inferences in terms of effect size. % from the relative effect size
% perspective
The recalibration is chosen such that for borderline significant original
studies, replication success can only be achieved if the replication effect
estimate is larger than the original one. The third extension provides a
framework for Bayesian design of replication studies. The framework allows
combining the data from the original study with external knowledge, which leads
to potentially more efficient designs compared to classical approaches.

The second part of the thesis is concerned with reverse-Bayes approaches in
general. The reverse-Bayes idea was first proposed in the 1950s, but it has
mostly been forgotten. To increase awareness and show potential use cases,
reverse-Bayes history and methods are summarized in a comprehensive review.
Furthermore, a short commentary on a recently proposed reverse-Bayes method
draws connections to other reverse-Bayes methods.

The last part of the thesis revolves around research integrity issues in
methodological research. Questionable research practices, such as selective
reporting of results, are often seen as main cause for the low replicability in
applied research. These practices can similarly harm methodological research,
but this is often not recognized. To raise awareness, an illustrative simulation
study is conducted in which it is shown how a novel method can easily be
presented as superior over established competitor methods if questionable
research practices are employed.


\textbf{Key words}: Bayesian inference, meta-science, replication studies


\newpage ~

\newpage



\begin{center}
  \vspace*{5cm}

  \textit{Dedicated to my mother Christa. \\
    You are deeply missed.}
\end{center}
