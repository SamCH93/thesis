\newpage
\section{Thesis summary}
\label{sec:summary}
\raggedbottom

This thesis consists of six papers. Paper I and II (and to a lesser extent III)
focus on extending the reverse-Bayes assessment of replication success from
\citet{Held2020}. Paper III presents a general Bayesian framework for
replication study design. Paper IV is a review paper about reverse-Bayes
methodology. Paper V is a short comment on another article which proposed a
reverse-Bayes method. Paper VI lists and illustrates questionable research
practices in simulation studies.


\startPaperSummary{PaperI}{%
  The sceptical Bayes factor for the assessment of replication success}{%
  Samuel Pawel, Leonhard Held \\
  \textit{Journal of the Royal Statistical Society: Series B (Statistical
    Methodology)}, 2022, 84(3), 879--911.
  \url{https://doi.org/10.1111/rssb.12491}.}

The reverse-Bayes approach from \citet{Held2020} is based on challenging the
original study with a sceptical prior so that there is no longer evidence for an
effect. Evidence is quantified in terms of credible intervals, respectively,
tail-probabilities. However, there exist also other measures of evidence, and it
has been a matter of long debates which is the most appropriate \citep[see
\eg{}][]{Berger1987, Casella1987, Royall1997, Berger2003, Benjamin2017,
  Lakens2018, Amrhein2019b}. In this paper, we therefore extend the
reverse-Bayes assessment of replication studies to use Bayes factors
\citep{Good1958, Jeffreys1961} for quantifying evidence and prior-data
conflict. % Bayes factors are also
% used for prior-data conflict assessment as suggested by \citep{Box1980}.
Similarly to the sceptical $p$-value, the procedure leads to a single measure
for quantifying the degree of replication success, \emph{the sceptical Bayes
  factor}. Systematic comparisons show that the sceptical Bayes factor shares
most properties with the sceptical $p$-value due to the reverse-Bayes approach
underlying both methods, yet in some situations conclusions may also differ
because of their different ways of quantifying evidence. Specifically, it is
shown that the sceptical $p$-values suffers from a certain type of ``shrinkage
paradox'', which is avoided by the sceptical Bayes factor; when the $p$-value
from the original study goes to zero, replication success based on the sceptical
$p$-value can be achieved with any arbitrarily small replication effect
estimate, whereas replication success based on the sceptical Bayes factor poses
a finite limit on how much shrinkage is allowed. Technically, the procedure is
more involved and closed-form solutions for the sceptical Bayes factor are only
available in special situations. % and in terms of the
% ``Lambert W function'' \citep{Corless1996}.
The method is illustrated on data from the Social Sciences Replication Project
\citep{Camerer2018}, and implemented in an R package.

The idea to use Bayes factors instead of tail probabilities was suggested by
\citet{Consonni2019} and \citet{Pericchi2020} independently in response to
original article by \citet{Held2020}. L. Held then implemented a first version
of the procedure for the grant application of this research project
\citep{Heldproposal2019}. I then worked out the technical and implementation
details, including closed-form solution for the sceptical Bayes factors,
asymptotic properties, type-I and type-II error rates, and non-normal extensions
of the procedure. I wrote the initial draft of the manuscript and the R package.
Throughout, L. Held gave high-level feedback. I presented initial results at the
GMDS and CEN-IBS conference in 2020 (online), L. Held presented the final
results at the ISBA world meeting 2021 (online).

\startPaperSummary{PaperII}{%
  The assessment of replication success based on relative effect size}{%
  Leonhard Held, Charlotte Micheloud, Samuel Pawel \\
  \textit{The Annals of Applied Statistics}, 2022, 16(2), 706--720.
  \url{https://doi.org/10.1214/21-AOAS1502}.}

It is not clear how to numerically interpret the sceptical $p$-value, as it is
not an ordinary $p$-value (which has a uniform distribution under the
corresponding null hypothesis). Similarly, it is unclear which threshold should
be used in case the sceptical $p$-value needs to be dichotomized into
replication success/failure. In this article, we therefore look closer at the
``success region'' of the sceptical $p$-value in terms of the relative effect
estimate $d = \that_{r}/\that_{o}$. This perspective leads to the proposal of a
new default level for thresholding the sceptical $p$-value called the
\emph{golden level} $\alpha_{G}$ (because the golden ration appears in its
derivation). The golden level is defined through the property that for an
original study which was just borderline significant ($p_{o} = \alpha$),
replication success based on $p_{\text{S}} \leq \alpha_{G}$ is only possible if
the replication effect estimate is at least as large as the original one
($d \geq 1$). The behavior of the golden level seems to align with common sense;
For original studies which were already convincing (in terms of their $p$-value)
the effect estimate in the replication study is allowed to shrink, to some
extent, whereas for less convincing original studies (those with $p$-values
around the significance level) shrinkage is more strongly penalized. We find
that in typical situations, replication success based on the golden level also
has similar or improved frequentist properties (type-I error rate and project
power) compared to the standard significance criterion. Case studies from four
large-scale replication projects illustrate the properties of the method.

L. Held had the idea to apply the sceptical $p$-value method to the data from
the four replication projects, which I collected for my master thesis. C.
Micheloud and L. Held came up with the golden level. L. Held wrote an initial
draft of the manuscript. C. Micheloud, L. Held, and I then iteratively worked on
the manuscript.

\startPaperSummary{PaperIII}{%
  Bayesian approaches to designing replication studies}{%
  Samuel Pawel, Guido Consonni, Leonhard Held \\
  2022. arXiv preprint.
  \url{https://doi.org/10.48550/arXiv.2211.02552}.}

An important aspect in the design of replication studies is determining their
sample size. How exactly the sample size should be determined depends on the
method which will be used for analyzing the replication data. Various approaches
have been proposed for doing so which are specifically tailored to certain
analysis methods. In this article, we provide a general Bayesian framework which
applies to any analysis method (including the sceptical $p$-value and the
sceptical Bayes factor). We show how the data from the original study and
external knowledge can be combined in a \emph{design prior} for the underlying
model parameters. Based on a design prior, predictions about the replication
data can then be computed, and the replication sample size can be chosen such
the probability of replication success becomes as high as desired. We illustrate
Bayesian design of replication studies in the normal-normal hierarchical model
which provides sufficient flexibility for specification of design priors. Data
from a cross-laboratory replication project are used for illustrating our
methods, which are also available in an R package.

L. Held specified in the grant application of this research project
\citep{Heldproposal2019} that we will investigate power and sample size
calculations for the sceptical $p$-value and the sceptical Bayes factor. For the
first paper I already derived the power function of the sceptical Bayes factor
in closed-form for two types of design priors. After its completion, I
generalized the result to any design prior in the normal-normal hierarchical
model, and started working on this manuscript. I presented a first draft to L.
Held and G. Consonni in the beginning of 2021. G. Consonni then helped
developing the methodology for multisite replication study design. I continued
working on the manuscript in 2022 and also wrote the accompanying R package.
Throughout, L. Held and G. Consonni gave high-level feedback.


\startPaperSummary{PaperIV}{%
  Reverse-Bayes methods for evidence assessment and research synthesis}{%
  Leonhard Held, Robert Matthews, Manuela Ott, Samuel Pawel \\
  \textit{Research Synthesis Methods}, 2022, 13(3), 295--314.
  \url{https://doi.org/10.1002/jrsm.1538}.}

While the popularity of Bayesian methods has been rapidly increasing since the
advent of modern computational methods in the 1990s, reverse-Bayes methods % , such
% as the reverse-Bayes assessment of replication studies,
have remained largely unknown to statisticians and users of statistics alike. In
this article, we review reverse-Bayes history and methods to increase awareness
about the approach. Specifically, we summarize the work on reverse-Bayes by I.
J. Good \citep{Good1950}, who first proposed the idea. We then review methods
such as the \emph{Analysis of Credibility} from \citet{Matthews2001b,
  Matthews2018}, its extension to Bayes factors, and the \emph{False Positive
  Risk} from \citet{Colquhoun2017}. To illustrate these method, we use data from
a meta-analysis on the effect of corticosteroids on COVID-19 mortality.

L. Held and M. Ott started working on this article several years ago when M. Ott
was still a PhD student (around 2017). When I discovered the connection between
the Analysis of Credibility and the fail-safe N method, L. Held suggested that
it would fit nicely into this manuscript, and that I should start to overhaul
it. I rewrote and expanded his initial draft, adding also a new section on
reverse-Bayes approaches with Bayes factors, largely based on the work from
paper I. We then managed to recruit R. Matthews to also contribute. From that
point on the three of us iteratively worked on the manuscript and M. Ott gave
high-level feedback.

\startPaperSummary{PaperV}{%
  Comment on ``Bayesian additional evidence for decision making under small
sample uncertainty''}{%
   Samuel Pawel, Leonhard Held, Robert Matthews \\
  \textit{BMC Medical Research Methodology}, 2022, 22(149).
  \url{https://doi.org/10.1186/s12874-022-01635-4}.}

Shortly after the acceptance of paper III, the article by \citet{Sondhi2021}
appeared. It proposed a novel reverse-Bayes method called \emph{Bayesian
  Additional Evidence}, and we noted some flaws in the article. This prompted us
to write a commentary. We show that --contrary to the statement by
\citet{Sondhi2021}-- there is a closed form solution for the key quantity in
their approach termed ``Bayesian Additional Evidence tipping point''.
% The Bayesian Additional Evidence method provides an answer to the question
% ``What prior do I have to choose so that that my unconvincing data become
% convincing?''.
The method is also closely related the Analysis of Credibility by
\citet{Matthews2018}. We investigate differences and similarities of the two
methods, showing that the priors determined through the Bayesian Additional
Evidence method are not always helpful.

R. Matthews attended us about the article from \citet{Sondhi2021}. After reading
it, I realized that their statement about closed-form solutions was incorrect
and derived a solution. L. Held suggested to write a commentary. I wrote an
initial draft of the manuscript, which R. Matthews substantially improved. The
two of us iteratively worked on the manuscript, while L. Held gave mostly
high-level feedback.

\startPaperSummary{PaperVI}{%
  Pitfalls and Potentials in Simulation Studies}{%
  Samuel Pawel, Lucas Kook, Kelly Reeve \\
  2022. arXiv preprint.
  \url{https://doi.org/10.48550/arXiv.2203.13076}.}

Simulation studies are frequently used for evaluating statistical methods.
However, several studies showed that the reporting standards in simulation
studies have remained low over the years \citep{Hoaglin1975, Burton2006,
  Morris2019}. Moreover, some authors have recently argued that also
methodological research is suffering from reproducibility issues, publication
bias, and a ``replication crisis'' due to researchers engaging in questionable
research practices, such as selective reporting \citep{Boulesteix2020}. In this
article, we raise awareness about these issues. We summarize possible
questionable research practices in simulation studies, and show how easy it is
make a method seem superior if various questionable research practices are
employed. We also give recommendations which could help to alleviate these
issues, most importantly we recommend researchers to write and pre-register
simulation protocols.

The manuscript is co-first authored by myself and L. Kook. I had the idea to
invent a ``mock-method'' and use questionable research practices that make it
seem superior, to draw attention to the low standards in methodological
research. I then wrote a first draft of the manuscript and proposed the idea to
L. Kook and K. Reeve. L. Kook and myself then came up with the method ``AINET''
and started writing the simulation protocol. L. Kook took lead in developing the
R package and simulation study code. K. Reeve provided feedback on the
simulation protocol and helped polishing the manuscript. Recently, I was invited
to present the results from this project at the CEN conference 2023 in Basel.
