% % Aus der Promotionsverordnung MNF §11 (4):
% % "Dem wissenschaftlichen Teil der Dissertation ist eine allgemein
% % verständliche kurze Zusammenfassung voranzustellen."

% \cleardoublepage
% \pdfbookmark{Zusammenfassung}{abstract_DE}

% { % need this extra grouping to avoid carry-over effects
% \begin{otherlanguage}{ngerman}
% \enlargethispage{4\baselineskip}
% \begin{center}\abstractnamefont
%   Lorem ipsum dolor sit amet, consectetur adipiscing elit, sed do eiusmod tempor
%   incididunt ut labore et dolore magna aliqua. Ut enim ad minim veniam, quis
%   nostrud exercitation ullamco laboris nisi ut aliquip ex ea commodo consequat.
%   Duis aute irure dolor in reprehenderit in voluptate velit esse cillum dolore
%   eu fugiat nulla pariatur. Excepteur sint occaecat cupidatat non proident, sunt
%   in culpa qui officia deserunt mollit anim id est laborum. Lorem ipsum dolor
%   sit amet, consectetur adipiscing elit, sed do eiusmod tempor incididunt ut
%   labore et dolore magna aliqua. Ut enim ad minim veniam, quis nostrud
%   exercitation ullamco laboris nisi ut aliquip ex ea commodo consequat. Duis
%   aute irure dolor in reprehenderit in voluptate velit esse cillum dolore eu
%   fugiat nulla pariatur. Excepteur sint occaecat cupidatat non proident, sunt in
%   culpa qui officia deserunt mollit anim id est laborum. Lorem ipsum dolor sit
%   amet, consectetur adipiscing elit, sed do eiusmod tempor incididunt ut labore
%   et dolore magna aliqua. Ut enim ad minim veniam, quis nostrud exercitation
%   ullamco laboris nisi ut aliquip ex ea commodo consequat. Duis aute irure dolor
%   in reprehenderit in voluptate velit esse cillum dolore eu fugiat nulla
%   pariatur. Excepteur sint occaecat cupidatat non proident, sunt in culpa qui
%   officia deserunt mollit anim id est laborum.

% \end{center}
% \end{otherlanguage}
% }


%%%%%%%%%%%%%%%%%%%%%%%%%%%%%%%%%%%%%%%%%%%%%%%%%%%%%%%%%%%%%%%%%%%%%%%%

\cleardoublepage
\pdfbookmark{Abstract}{abstract_EN}

\enlargethispage{4\baselineskip}
\begin{center}\abstractnamefont
  \textbf{Abstract}
\end{center}

% The credibility of a scientific finding is based on the fact that it can be
% replicated when the same study is repeated with new subjects.
% An important aspect of the credibility of a scientific finding is that the
% finding can be replicated when the same study is repeated with new subjects.
% An important aspect of the credibility of a scientific finding is its
% replicability -- the ability to produce similar results when the same study is
% repeated with new subjects.
An important aspect of the credibility of scientific findings is their
replicability -- the ability that a similar finding can be obtained when the
same study is repeated with new subjects. However, various failures to replicate
major scientific findings in the social and life sciences indicate that
replicability is often lower than expected. This ``replication crisis'' has led
to several methodological reforms in the past decade, an increased conduct of
replication studies being one of them. Despite this increase in replication
studies, there is no consensus on a fundamental question: How should replication
success be defined statistically?
% Various methods already exist and various new ones have been proposed in
% response to the crisis. This thesis centres around a proposal based on the
% reverse-Bayes approach. The key idea is to reverse the traditional ``forward''
% use of Bayes' theorem (``prior + likelihood $\rightarrow$ posterior''),
% starting instead with a pre-specified posterior and deducing the corresponding
% prior (``posterior + likelihood $\rightarrow$
% prior''). % As such, the reverse-Bayes approach naturally fits to the replication
% % setting as it enables the mathematical formalization of scepticism regarding the
% % finding from the original study.
% This allows us to challenge the original finding by determining a sceptical
% prior for the underlying effect size such that the resulting posterior is no
% longer convincing. This prior then represents the position of a sceptic who
% remains unconvinced by the original study. Whether or not the sceptics'
% position is justified can then be assessed in light of the new data from the
% replication study. The larger the conflict between the replication data and
% the sceptic, the larger the degree of replication success. This procedure can
% be summarized in a single quantitative measure of replication success, termed
% the sceptical $p$-value.

Possible answers to this question are given in the first and largest part of
this thesis. It consists of three extensions to a recently proposed
reverse-Bayes method for quantifying replication success. The key idea of the
method is to challenge the data from the original study with a sceptical prior
distribution so that the resulting posterior distribution no longer indicates
evidence for an effect. The goal of the replication study is then to show that
the sceptical prior is unrealistic, and replication success is achieved if there
is sufficient conflict between the prior and the replication data. The procedure
can be summarized in a single quantitative measure, termed the sceptical
$p$-value.
% extension 1
The first extension recalibrates the sceptical $p$-value so that replication
success takes effect size more appropriately into account. Specifically, the
recalibration is chosen such that for original studies which were borderline
significant, replication success can only be achieved if the effect estimate
from the replication study is larger than the effect estimate from the original
study. We find that the recalibrated sceptical $p$-value also has good
frequentist properties comparable to the standard method used in practice.
% extension 2
The second extension replaces tail probabilities with Bayes factors as measures
of evidence. The procedure can again be summarized in a single measure called
the sceptical Bayes factor.
% The sceptical prior is now determined such that the original finding is no
% longer convincing in terms of a Bayes factor. In contrast to tail
% probabilities,
% We find that the extended method combines two notions of replicability: It
% requires that original and replication study independently provide evidence for
% an effect and ensures that their effect estimates are compatible.
The sceptical Bayes factor has similar properties as the sceptical $p$-value but
it avoids a statistical paradox which the sceptical $p$-value suffers from -- in
some situations the sceptical $p$-value may indicate replication success even
though the effect estimate from the replication is arbitrarily smaller than the
effect estimate from the original study. % , which cannot happen with the sceptical
% Bayes factor.
% Bayes factors have a more
% natural interpretation and allow for direct quantification of evidence for one
% hypothesis versus another. As with the sceptical $p$-value, the procedure leads
% to a single measure quantifying the degree of replication success called the
% sceptical Bayes factor.
The third extension provides a framework for Bayesian design of replication
studies which allows for combining data from the original study with external
knowledge. This approach can lead to potentially more efficient designs compared
to classical approaches, and it can be used for replication design based on both
the sceptical $p$-value and the sceptical Bayes factor.
% External knowledge, such as data from other studies, is ubiquitous in the
% replication setting and taking it into account can lead to potentially more
% efficient designs compared to classical approaches.

The second part of this thesis also deals with reverse-Bayes methods, but with a
broader scope of applications than replication studies. The reverse-Bayes idea
was first proposed in the 1950s, but it has mostly been forgotten. To increase
awareness and show potential use cases, reverse-Bayes history and methods are
summarized in a comprehensive review. The review includes also some new results
on connections between reverse-Bayes methods and meta-analysis. % Finally, a
% commentary paper on a recently proposed reverse-Bayes method draws connections
% to other reverse-Bayes methods.

The last part of this thesis takes a meta-scientific perspective on
methodological research. Questionable research practices, such as selective
reporting of results, are often seen as main cause for replicability issues in
the medical and social sciences. These practices can similarly harm
methodological research, but are often not recognized. To raise awareness, an
illustrative simulation study is conducted in which it is shown how a novel
method can easily be presented as superior over established competitor methods
if questionable research practices are employed. Finally, several
recommendations are given to alleviate these issues.


\textbf{Key words}: Bayesian inference, meta-science, replication studies


\newpage ~

\newpage



\begin{center}
  \vspace*{5cm}

  \textit{Dedicated to my mother Christa. \\
    You are deeply missed.}
\end{center}
