% % Aus der Promotionsverordnung MNF §11 (4):
% % "Dem wissenschaftlichen Teil der Dissertation ist eine allgemein
% % verständliche kurze Zusammenfassung voranzustellen."

\cleardoublepage
\pdfbookmark{Abstract}{abstract_EN}

\enlargethispage{4\baselineskip}
\begin{center}\abstractnamefont
  \textbf{Abstract}
\end{center}

An important aspect of the credibility of scientific findings is their
replicability -- the ability that a similar finding can be obtained when a study
is repeated with new subjects. However, various failures to replicate major
scientific findings in the social and life sciences indicate that replicability
is often lower than expected. This ``replication crisis'' has led to several
methodological reforms in the past decade, an increased conduct of replication
studies being one of them. Despite this increase in replication studies, there
is no consensus on a fundamental question: How should replication success be
defined statistically?

Possible answers to this question are given in the first and largest part of
this thesis. It consists of three extensions to a recently proposed
reverse-Bayes method for quantifying replication success. The methods' key idea
is to challenge the data from the original study with a sceptical prior
distribution for the effect size so that the resulting posterior distribution no
longer indicates evidence for an effect. The goal of the replication study is
then to show that the sceptical prior is unrealistic, and replication success is
achieved if there is sufficient conflict between the prior and the replication
data. The procedure can be summarized in a single quantitative measure, termed
the sceptical $p$-value.
% extension 1
The first extension recalibrates the sceptical $p$-value so that replication
success takes effect size more appropriately into account. Specifically, the
recalibration is chosen such that for original studies which were borderline
significant, replication success can only be achieved if the effect estimate
from the replication study is larger than the effect estimate from the original
study. We find that the recalibrated sceptical $p$-value also has good
frequentist properties comparable to the standard method used in practice.
% extension 2
The second extension replaces tail probabilities with Bayes factors as measures
of evidence. The procedure can again be summarized in a single measure called
the sceptical Bayes factor. The sceptical Bayes factor has similar properties as
the sceptical $p$-value but it avoids a statistical paradox which the sceptical
$p$-value suffers from -- in some situations the sceptical $p$-value may
indicate replication success even though the effect estimate from the
replication is substantially smaller than the effect estimate from the original
study.
% extension 3
The third extension provides a framework for Bayesian design of replication
studies which allows for combining data from the original study with external
knowledge. This approach can lead to potentially more efficient designs compared
to classical approaches, and it can be used for replication design based on both
the sceptical $p$-value and the sceptical Bayes factor.

The second part of this thesis also deals with reverse-Bayes methods, but with a
broader scope of applications than replication studies. The reverse-Bayes idea
was first proposed in the 1950s, but it has mostly been forgotten. To increase
awareness and show potential use cases, reverse-Bayes history and methods are
summarized in a comprehensive review. The review includes also some new results
on connections between reverse-Bayes methods and meta-analysis. % Finally, a
% commentary paper on a recently proposed reverse-Bayes method draws connections
% to other reverse-Bayes methods.

The last part of this thesis takes a meta-scientific perspective on
methodological research. Questionable research practices, such as selective
reporting of results, are often seen as main cause for replicability issues in
the medical and social sciences. These practices can similarly harm
methodological research, but are often not recognized. To raise awareness, an
illustrative simulation study is conducted in which it is shown how a novel
method can easily be presented as superior over established competitor methods
if questionable research practices are employed. Finally, several
recommendations are given to alleviate these issues.


\textbf{Key words}: Bayesian inference, meta-science, replication studies


\newpage ~

\newpage



\begin{center}
  \vspace*{5cm}

  \textit{Dedicated to my mother Christa. \\
    You are deeply missed.}
\end{center}
